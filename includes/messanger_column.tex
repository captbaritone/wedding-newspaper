\byline{\Large ``...That Which We Call A Rose''}{Stephen Messenger}
Throughout history and in cultures around the world, mankind has held a deep
connection with flowers. From the smallest blossoms emerging from the melting
snow, marking the end of winter, to elaborate bouquets given as a gesture of
love, flowers are unmatched in their ability to please the senses and delight
the soul.

Flowering plants, unlike their fruit and vegetable bearing counterparts, have
been infused with symbolism that transcends their colorful bloom. A poetic
regard for flowers is evident even in Neanderthal culture with the discovery of
burial sites containing Hollyhock flowers---an indication that they too
considered it as ``holy'' as its name continues to suggests today.

It is not surprising then, that the names by which we refer to flowers reflect
a loftier esteem than that given to, say, broccoli, which derives its rather
unappetizing moniker from the Italian {\it brocco}, meaning simply a ``shoot'' or
``stalk'' in line with opinion held by countless picky eaters. Indeed, the
names given to flowers denote their benefit to the spirit, richly steeped in
culture and tradition.

Like many words in our language, flowers' names hold clues about their history
and relationship with us. The daisy, for example, known for its small yellow
blossoms, is quiet common throughout the world. Daisies are unique in that they
close their golden petals during the night and keep them shut, as if in sleep,
until the morning. This peculiar characteristic so earned this little flower
the name ``day's eye'' from speakers of Old English, which was eventually
compounded into the form we use today.

Dandelions also derive their name from their characteristically numerous thick
and slender yellow petals. It is not so difficult for an imaginative observer
to equate the Dandelion's coarse petals to rows of teeth on some well-fanged
beast, a comparison which explains its French origin {\it dent de lion}, or in
English ``teeth of a lion.''

Some flowers, on the other hand, were named, not from their appearance alone,
but for their association with mythology. The Iris, a flower which appears in
a wide variety of colors, shares its name with the Greek goddess who unified
heaven and earth, personified by, aptly enough, the rainbow. The Narcissus
flower, too, is said to have sprouted upon the death of its namesake, though
there is no evidence that the flower is as self-absorbed. The Virgin Mary also
has left a mark on floral taxonomy with the Marigold, or Mary's Gold, which,
according to Flemish tradition, sprouted from her tears.

Still, other flowers' names offer some insight into how they had some
usefulness in the past. The sweet, aromatic Lavender was used to add a pleasant
scent to recently washed clothes and to perfume bathwater---as evidenced by its
association with the Latin lavare, meaning ``to wash.'' Calluna, the flowering
shrub also known as Heather, seems to have been appreciated not so much for its
beauty as its handiness as a broom---its name originating from the Greek word
for ``to sweep.''

The Pansy blossom, which in late summer begins to droop, is perhaps the most
thoughtful of flowers, its name derived from the French pensée, or
``thought''---the source for the English word pensive. Carnations are also
appreciated for their human qualities, their often soft pink petals likened to
the hue of skin, sharing its meaning closely with the word incarnation, ``to be
made flesh,'' (though some early writings refer to this flower as coronation,
which some scholars believe is an allusion to its use as a garland in Greek
tradition).

Flowers are so unique in their seemingly universal appeal that not only do they
bear names that reflect their colorful history, but nearly every culture names
their children after them. From Ambuj (Indian for lotus) to Zara (Arabic for
a blossom), floral names are timeless in their popularity. While millions of
people around the world share their names with flowers, the opposite is also
true. The Zinnia and Dahlia flowers, for example, can thank the 18th-century
botanists Johann Zinn and Anders Dahl, respectively, for their names.

Ultimately, the names assigned to flowers are less of a reflection on the
flowers themselves, and more on the long-standing relationship of love and
esteem we have for them. So, while if a flower were in fact known by
a different name it would indeed smell as sweet, the stories behind their names
would remain human stories---our contribution to one of nature's most cherished
creations.
