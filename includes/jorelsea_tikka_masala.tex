\byline{\Large Chicken Ma(r)sala}{Chelsea Hollow and Jordan Eldredge}
If there is one great proverb I have seen in action in the Hollow Family it is:
``The key to a man's heart is through his stomach!''

Jordan and I had been dating for a couple of weeks when he mentioned that his
favorite meal was Chicken Masala. ``Chicken Marsala?'' I replied, ``that is one of
my specialties!'' So we arranged a date at my house and I bought a nice bottle of wine. 

Masala is best if it doesn't sit so I thought we could prepare it together.
I asked Jordan to grab the mushrooms from the refrigerator to which he sweetly
and curiously replied, ``I've never had masala with mushrooms in it.'' Of course
I replied, ``Oh! The only way to make marsala is with mushrooms! It simply
doesn't taste the same without{\ldots}they marinate and bring out all of the
delicious flavors!'' As we continued he said, ``How do you get the sweetness? Do
you use butternut squash?'' I giggled, ``Squash? No, it's just the caramelization
from the onions and the sweetness from the marsala wine reduction.''

``There's wine in it?!''

Now at this point, it was pretty clear to us both that we had very different
ideas about what we were making. We aren't really sure how it finally came out
but we certainly got a great laugh out of the discovery that I was cooking the
traditional Italian Chicken Marsala and Jordan's favorite dish is the Indian
Chicken Tikka Masala!

Gratefully he loved mine and eventually we took on making the Tikka Masala for
our first day after Thanksgiving celebration at our house with both of our
families. We hope you enjoy these recipes!

\begin{recipe}{Chicken Tikka Masala} {4 servings} {9.5 hours}
\freeform Chicken
\ingredient[1]{kilogram}{skinless, boneless chicken thighs, fat trimmed}
\ingredient[\fr14]{cup}{plain yogurt}
\ingredient[]{}{salt (to taste)}
Cut chicken into pieces and marinate with yogurt and salt. Keep aside for 1 hour.
\ingredient[\fr14]{cup}{oil}
\ingredient[1]{teaspoons}{turmeric powder}
Heat oil and fry tumeric
\ingredient[1]{teaspoon}{ginger poweder}
\ingredient[1]{teaspoon}{garlic poweder}
Add ginger, garlic mixed in a tablespoon of water
\ingredient[1]{}{onion (finely chopped)}
\ingredient[2]{teaspoon}{chilli poweder}
\ingredient[2]{teaspoon}{coriander poweder}
Stir well. Add chilli and coriander in 4 tablespoons of water. Stir until oil is clear. Add chicken and enough water to make a good sauce. Cover and chook on gentle heat until chicken is cooked.

\freeform Masala Marinade
\ingredient[\fr14]{cups}{vegitable oil}
Heat until shimmering
\ingredient[3]{}{cardomom pods}
\ingredient[2]{}{cinnamon sticks}
\ingredient[1]{}{bay leaf}
Add spices, cook until sizzling (about three minutes)
\ingredient[1]{}{onion (large) diced small}
\ingredient[\fr12]{teaspoon}{salt}
\ingredient[1]{tablespoon}{garam masala} 
\ingredient[1]{teaspoon}{corriander ground}
\ingredient[1]{teaspoon}{paprika}
\ingredient[1]{teaspoon}{turmuric}
\ingredient[1]{teaspoon}{cayenne}
\ingredient[1]{teaspoon}{fresh ginger (minced)}
\ingredient[2]{cloves}{garlic}
Remove cardomom pods. Add spices, garlic and onion
\ingredient[\fr12]{cup}{water}
\ingredient[3\fr12]{14.5oz cans}{peeled diced tomatoes (drained)}
Add tomatoes and water. Cover partially and cook over moderate heat for 20 minutes.
\ingredient[1\fr14]{cups}{heavy cream}
Add cream and cook over low heat, stirring occasionally, until thickend (about
10 minutes). Stir in chicken; simmer gently for 10 minutes, stirring frequenty,
and server.
\end{recipe}

\begin{recipe}{Simple Rice} {4 servings} {30 minutes}
\ingredient[2]{cups}{white basmati rice}
\ingredient[4]{teaspoons}{butter}
\ingredient[1]{teaspoon}{mustard seed (whole)}
\ingredient[1]{teaspoon}{cumin (ground)}
\ingredient[3]{cups}{water}
\ingredient[1]{teaspoon}{salt (optional)}
\ingredient[\fr12]{teaspoon}{black pepper}
Melt butter in pan. Add spices and cook gently. Rinse rice and add to butter and spices. Cover with water and bring to boil. Let boil on low heat until wateris gone.
\end{recipe}

% http://www.foodnetwork.com/recipes/tyler-florence/chicken-marsala-recipe/index.html
\begin{recipe}{Chicken Marsala} {4 servings} {40 minutes}
\ingredient[4]{}{skinless, boneless chicken breasts}
\ingredient[]{}{flour, for dredging}
\ingredient[]{}{salt and pepper}
Put the chicken breasts side by side on a cutting board and lay a piece of plastic wrap over them; pound with a flat meat mallet, until they are about 1/4-inch thick. Put some flour in a shallow platter and season with a fair amount of salt and pepper; mix with a fork to distribute evenly.
\ingredient[\fr14]{cup}{olive oil}
Heat the oil over medium-high flame in a large skillet. When the oil is nice and hot, dredge both sides of the chicken cutlets in the seasoned flour, shaking off the excess. Slip the cutlets into the pan and fry for 5 minutes on each side until golden, turning once – do this in batches if the pieces don't fit comfortably in the pan. Remove the chicken to a large platter in a single layer to keep warm.
\ingredient[8]{ounces}{crimi or procini mushrooms (stemmed and halved)}
\ingredient[\fr12]{cup}{sweet Marsals wine}
Lower the heat to medium and add the mushrooms and saute until they are nicely browned and their moisture has evaporated, about 5 minutes; season with salt and pepper. Pour the Marsala in the pan and boil down for a few seconds to cook out the alcohol. 
\ingredient[\fr12]{cup}{chicken stock}
\ingredient[2]{tablespoons}{unsalted butter}
\ingredient[\fr14]{cup}{chopped flat-leaf parsley}
Add the chicken stock and simmer for a minute to reduce the sauce slightly. Stir in the butter and return the chicken to the pan; simmer gently for 1 minute to heat the chicken through. Season with salt and pepper and garnish with chopped parsley before serving.
\end{recipe}

